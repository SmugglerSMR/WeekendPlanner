%%%%%%%%%%%%%%%%%%%%%%%% CAB432 Group Report %%%%%%%%%%%%%%%%%%%%%%%

%BEGIN_FOLD
% Disclaimer: You HAVE to use this. This is a not simple starting point, all libraries and titles were recreated for fit group Assignment, same way it was in EGB342. If something does not work... Well, you all know who to blame.

% This sets the style of the document, you can use different built in styles, create your own .cls files or download ones from the Internet. This one is fairly standard to use
\documentclass[12pt]{article}
\usepackage[natbibapa]{apacite}


%%%%%%%%%%%%%%%%%%%%%%%%%%%%% Packages %%%%%%%%%%%%%%%%%%%%%%%%%%%%%%

% This package is handy for captioning figures, you can set caption style here as well
\usepackage[font={small,it}]{caption}
\usepackage[a4paper, portrait, margin=0.8in,top=2cm,bottom=2cm,]{geometry}
\usepackage{bigstrut}

% This is important for position images as latex will put your image where it best fits unless you tell it otherwise
\usepackage{float}
\usepackage{wrapfig}

% If you want images this is necessary
\usepackage{graphicx}
\usepackage{subcaption}
\graphicspath{{./images}}
\usepackage{hyperref}
\usepackage{color}
\usepackage{xcolor}
\hypersetup{colorlinks=true}
\hypersetup{linkcolor=blue}
\newcommand\PlaceText[3]{%
\begin{tikzpicture}[remember picture,overlay]
\node[outer sep=0pt,inner sep=0pt,anchor=south west] 
  at ([xshift=#1,yshift=-#2]current page.north west) {#3};
\end{tikzpicture}%
}
% You can use this to set your margin size
%\usepackage[margin=25mm]{geometry}

% Allows you to do things such as headers and footers
\usepackage{fancyhdr}

% This needs to be in here if you want to set up your document with more than one column in sections 
\usepackage{multicol}

% Here are a few packages that help with formatting equations, you may not need to use this but I find align* from amsmath particularly useful
\usepackage{amsmath,amssymb,amsthm,textcomp,amsfonts,amsthm,mathrsfs}
\usepackage{xparse}% http://ctan.org/pkg/xparse
\NewDocumentCommand{\ceil}{s O{} m}{%
  \IfBooleanTF{#1} % starred
    {\left\lceil#3\right\rceil} % \ceil*[..]{..}
    {#2\lceil#3#2\rceil} % \ceil[..]{..}
}

% Enhances Latex`s cross referencing
\usepackage{cleveref}
\usepackage{hyperref}
\hypersetup{colorlinks=true}
\hypersetup{linkcolor=blue}
\usepackage{xcolor}
\usepackage{physics}
\usepackage{gensymb}
\usepackage{mathrsfs}

% Also not necessary but I find it handy when formatting arrays and matrices
\usepackage{array}
\usepackage{xfrac}
\usepackage{enumitem}

%% These packages you`ll need to download a .sty file before you can use

% This allows really nice formatting for MATLAB code.
\usepackage[numbered,framed]{mcode}
\usepackage{mathrsfs}
\usepackage{hyperref}
\hypersetup{colorlinks=true}
\hypersetup{linkcolor=blue}
\usepackage{xcolor}
\usepackage{physics}
\usepackage{gensymb}

%% Feel free to add any more packages you want!!!
\usepackage{indentfirst}
\usepackage{parskip} 
\setlength\parindent{0pt}
%\setlength{\parskip}{1cm plus4mm minus3mm}
\usepackage{csquotes}
\usepackage{mathtools}
\newcommand{\Lim}[1]{\raisebox{0.5ex}{\scalebox{0.8}{$\displaystyle \lim_{#1}\;$}}}
\usepackage{wrapfig}

%%%%%%%%%%%%%%%%%%%%%%%%% Setup the document %%%%%%%%%%%%%%%%%%%%%%%%

\lstset{basicstyle=\scriptsize\ttfamily,breaklines=true}
\renewcommand{\thesubsection}{\thesection.\arabic{subsection}.}

\renewcommand{\thesubsubsection}{\indent \roman{subsubsection}}

\numberwithin{equation}{section} % Number equations within sections (i.e. 1.1, 1.2, 2.1, 2.2 instead of 1, 2, 3, 4)
\numberwithin{figure}{section} % Number figures within sections (i.e. 1.1, 1.2, 2.1, 2.2 instead of 1, 2, 3, 4)
\numberwithin{table}{section} % Number tables within sections (i.e. 1.1, 1.2, 2.1, 2.2 instead of 1, 2, 3, 4)

\newcommand{\horrule}[1]{\rule{\linewidth}{#1}} % Create horizontal rule command with 1 argument of height

\title{	
	\normalfont \normalsize 
	\textsc{Queensland University of Technology} \\ [25pt] 
	\horrule{0.5pt} \\[0.4cm] % Thin top horizontal rule
	\huge CAB432 - Assignment 1 – Mashup Project \\ Proposal \\ % The assignment title
	\author{ Marat (Matt) Sadykov \small n9312706 \\ \\ Tutor: Jacob Marks \small \underline{marksj@qut.edu.au}}
	\date{\normalsize\today} % Today`s date or a custom date
	\horrule{2pt} \\[0.5cm] % Thick bottom horizontal rule
}
%\title{\textbf{Assignment 1}\\ \large  EGB342 \\ Group 10}
%\author{Matt Sadov \small n9312706 \\ Stephen Hannam \small n2731061\\ Thomas Nugent \small n9219986}
%\date{May 2017}


% Headers and footers
\pagestyle{fancy}
\fancyhf{}
\rhead{Mashup Proposal}
\lhead{CAB432 Cloud Computing}
\cfoot{Marat (Matt) Sadykov}
\cfoot{Page \thepage}
\cfoot{n9312706}

%END_FOLD
%%%%%%%%%%%%%%%%%%%%% Begin the Actual Document %%%%%%%%%%%%%%%%%%%%%
\begin{document}
\maketitle
\newpage
\tableofcontents
\newpage
%===================================================%
%													%
%============ Section 1 Introduction ===============%
%												    %
%===================================================%
\section{Introduction}	
	\begin{quote}
		quoted text
	\end{quote}\label{ref:100}

	\begin{enumerate}
		\item Overall mashup purpose and description (1-2 Paragraphs)
		\item List of service and data APIs to be utilised. Short description of API (1 paragraph)
	\end{enumerate} 

	\subsection{World Map}
		\textcolor{blue}{blue colored text}\\
		\textcolor{red}{red colored text}
	
		\begin{lstlisting}
			function freq_tbl = wsTabulation(text_in)	
			end
		\end{lstlisting}
		\footnote{Footnote.}
		
		\begin{itemize}
			\item[] item1
			\item[] item2
		\end{itemize}
	
		\subsubsection{Google Maps}
			\begin{figure}[H]
				\centering
				\caption{Tough choice!}
				\includegraphics[width=0.6\textwidth]{images/UQ}
			\end{figure}		
			$$E[a_i] = \sum\limits_{i=1}^{20} p(a_i)L_i \  \textcolor{blue}{\approx 4.0958 \textsl{ bits per symbol}}$$ where $L_i$ is the bit-length of the binary code-word for symbol $a_i$.

			$\rightarrow$ 0, b $\rightarrow$ 01, c $\rightarrow$ 011.\hspace{1cm} $\leftarrow$ \underline{Uniquely decodeable (but not instantaneous).} \\ \\
		\subsubsection{Map Quest}
	\subsection{Flight Management}
		\subsubsection{Expedia}
	\subsection{Location Explorer}
		\subsubsection{WebCam}
		
%===================================================%
%													%
%============ Section 2 Use Cases ==================%
%												    %
%===================================================%
\newpage
\section{Use Cases}
	\subsection{Trip Planner}
	with several destinations and cost
	\begin{wrapfigure}[20]{r}{0.5\textwidth}
		\centering
		\includegraphics[width = 0.5\textwidth,]{images/image1.png}
		\caption{Constellation diagram for BPSK modulation scheme}
		\label{fig:BPSK_diagram}
	\end{wrapfigure}
	\subsection{Exploring area with cameras}
	\begin{align*}
		S_i(t) &= A \ cos(2\pi f_ct + \phi (t)) \\
		S_1(t) &= A \ cos(2\pi f_ct) \\ 
		S_2(t) &= A \ cos (2\pi f_ct+\pi)
	\end{align*}
	\subsection{Weekend planner}
	With cost calculation, surrounding exploration and etc.
%===================================================%
%													%
%============ Technical Breakdown===================%
%												    %
%===================================================%
\newpage
\section{Technical Breakdown}	
	\begin{enumerate}
		\item A clear division between Server, Client and Cloud.
		\item  a mock-up of your application page		
	\end{enumerate} 
	
	\begin{figure}[H]
		\centering		
		\includegraphics[width=\textwidth]{images/Breakdown}
		\caption{Data flow}
	\end{figure}
	
	
%	\begin{center}
%		\footnotesize\textit{i.e. a QPSK modualtion scheme}\normalfont
%	\end{center}
%
%	\begin{center}
%		\begin{tabular}{c}
%			\begin{lstlisting}[linewidth = 15cm, name = Thomas Nugent]
%				function [t, QPSK_msg, msgi, msgq, f] = QPSK_transmit(msg, start_time,...
%				Rs, Ns, A, fc)
%				\end{lstlisting}
%		\end{tabular}
%	\end{center}
%%===================================================%
%%													%
%%================= Difficulties ====================%
%%												    %
%%===================================================%
%\section{Difficulties}
%	\cite[p.~278]{GOD} \\ 
%	\cite{michelson}
%	
%	\begin{align}
%		S_{m}(t) = A_{mc}g(t)cos(2\pi f_{c}t) - A_{ms}g(t)sin(2\pi f_{c}t)
%	\end{align}
%	
%	\cite{fundcomm} This causes 
%
%	
%	\hyperref[pros_cons]{subsection further}.
%	
%	\begin{figure}[H]
%		\begin{subfigure}[b]{0.52\textwidth}
%			\centering
%			\includegraphics[width=\linewidth]{images/marie-b}
%			\caption*{8-QAM -5dB to -2dB.}
%			\label{fig:Xana}
%		\end{subfigure}
%		\begin{subfigure}[b]{0.52\textwidth}
%			\centering
%			\includegraphics[width=\linewidth]{images/Shelob}
%			\caption*{8-QAM -5dB to -2dB.}
%			\label{fig:Shelob}
%		\end{subfigure}
%		\caption{8-QAM -5dB to 16dB.}
%		\label{fig:My laptop}
%	\end{figure}
%	
%	\begin{equation}
%		Pb = \frac{4}{log_{2}M}Q\left\lbrace \sqrt{ \left( \dfrac{3(log_{2}M)}{M-1}\frac{Eb}{No}\right)}\right\rbrace
%	\end{equation}
%
%	\subsection{Subsection with label} \label{pros_cons}
%		To justify the difference with higher order schematics, it will be better to actually 	
%	
%		\begin{figure} [H]
%			\centering
%			\begin{minipage}[b]{0.45\textwidth}
%				\centering
%				\includegraphics[width=\linewidth]{images/image1}
%				\caption*{8-QAM 12dB to 16dB.}
%				\label{fig:16-QAM}	
%			\end{minipage}
%			\hfill
%			\begin{minipage}[b]{0.45\textwidth}
%				\centering
%				\includegraphics[width=\linewidth]{images/marie-b}
%				\caption*{8-QAM -4dB to 2dB.}
%				\label{fig:32-QAM}	
%			\end{minipage}
%			\begin{minipage}[b]{0.45\textwidth}
%				\centering
%				\includegraphics[width=\linewidth]{images/Shelob}
%				\caption*{8-QAM 12dB to 16dB.}
%				\label{fig:64-QAM}	
%			\end{minipage}
%			\hfill
%			\begin{minipage}[b]{0.45\textwidth}
%				\centering
%				\includegraphics[width=\linewidth]{images/UQ}
%				\caption*{8-QAM -4dB to 2dB.}
%				\label{fig:128-QAM}	
%			\end{minipage}
%			\caption{8-QAM -4dB to 2dB Main label.}
%			\label{fig:QAM-schemes}
%		\end{figure}
%%===================================================%
%%													%
%%================= Extensions ======================%
%%												    %
%%===================================================%
%\newpage
%\section{Extensions}
%
%%===================================================%
%%													%
%%=================== Testing =======================%
%%												    %
%%===================================================%
%\newpage
%\section{Testing}
%
%%===================================================%
%%													%
%%===================REFLECTIOM======================%
%%												    %
%%===================================================%
%\newpage
%\section{Reflection}
%Reflection
%
%%===================================================%
%%													%
%%===================Apeendix A======================%
%%												    %
%%===================================================%
%\newpage
%\section{Appendix A}
%Reflection
%
%%===================================================%
%%													%
%%============Bibliography and refferencing==========%
%%												    %
%%===================================================%
\begin{flushleft}
	\bibliographystyle{apacite}
	\bibliography{referencing/referenceList}
\end{flushleft}

\end{document}
